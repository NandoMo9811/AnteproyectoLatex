% ================================================================================================ %
% Anteproyecto de trabajo de grado - FIET Univerisidad del Cauca
% ================================================================================================ %

\documentclass[12pt,letterpaper,onecolumn]{IEEEtran}

% ------------------------------------------ %
% Idioma y Codificación
% ------------------------------------------ %
\usepackage[spanish]{babel}
\usepackage[utf8]{inputenc}
\usepackage[T1]{fontenc}

% ------------------------------------------ %
% Uso de Times New Roman
% ------------------------------------------ %
\usepackage{newtxtext,newtxmath}

% ------------------------------------------ %
% Margenes
% ------------------------------------------ %
\usepackage[
    left=3cm,
    right=3cm,
    top=3cm,
    bottom=3cm
    ]{geometry}

% ------------------------------------------ %
% Interlineado y espaciado
% ------------------------------------------ %
\usepackage{setspace}
\onehalfspacing

\setlength{\parskip}{6pt}

% ------------------------------------------ %
% Paquetes adicionales
% ------------------------------------------ %
\usepackage{ragged2e}
\usepackage{tocloft}
\usepackage{graphicx}
\usepackage{tabularx}
\usepackage{float}
\usepackage{array}
\usepackage{longtable}
\usepackage{caption}
%\usepackage[table]{xcolor}
\usepackage{multirow}
\usepackage{booktabs}
%\usepackage{amsmath}
\usepackage[hidelinks]{hyperref}
\usepackage{titlesec}

% ------------------------------------------ %
% Ajuste de Tabla de Contenidos y Otros
% ------------------------------------------ %
\addto\captionsspanish{
    \renewcommand{\contentsname}{TABLA DE CONTENIDO}
    }

\addto\captionsspanish{
    \renewcommand{\tablename}{Tabla}
    }
    
\makeatletter
\renewcommand{\tableofcontents}{
    \begin{center}
        \Large\bfseries TABLA DE CONTENIDO
    \end{center}
    \vspace{1em}
    \@starttoc{toc}
}
\makeatother

\renewcommand{\thesection}{\arabic{section}}
\renewcommand{\thesubsection}{\thesection.\arabic{subsection}}
\renewcommand{\thesubsubsection}{\thesubsection.\arabic{subsubsection}}
\renewcommand{\thetable}{\arabic{table}}


\titleformat{\section}
  {\Large\bfseries}{\thesection}{1em}{}

% ------------------------------------------ %
% Información del documento
% ------------------------------------------ %
\title{
    Diseño e Implementación de una Aplicación Móvil con Retroalimentación Háptica y Sonora para
    la Reducción de la Nomofobia en Jóvenes Adultos
}

\author{
    \IEEEauthorblockN{
        Maria Jose Cabrera Pantoja \and
        Fernando Molina Plaza
    }
    \IEEEauthorblockA{
        Universidad del Cauca\\
        Facultad de Ingeniería en Electrónica y Telecomunicaciones\\
        Correos: mcabrerap@unicauca.edu.co, fmolinap@unicauca.edu.co
    }
}

% ------------------------------------------ %
% Documento
% ------------------------------------------ %
\begin{document}

\begin{titlepage}
    \thispagestyle{empty}
    \begin{center}

        \vspace*{2cm}

        {\LARGE \textbf{Diseño e Implementación de una Aplicación Móvil con Retroalimentación 
        Háptica y Sonora para la Prevención de la Nomofobia en Jóvenes Adultos}}

        \vspace{2cm}

        \includegraphics[width=4cm]{figuras/LogoUniCauca.png}

        \vspace{2cm}

        {\large
        \textbf{Autores:}\\
        Maria Jose Cabrera Pantoja\\
        Fernando Molina Plaza}

        \vspace{1.5cm}

        {\large
        Director: PhD. César Alberto Collzos Ordóñez \\
        Codirector: Łukasz Tomczyk (Universidad Jaguelónica)
        }
        
        \vfill

        {\large
        Universidad del Cauca \\
        Facultad de Ingeniería Electrónica y Telecomunicaciones \\
        Departamento de Sistemas \\
        Popayán, Enero de 2026

        }

    \end{center}
\end{titlepage}

\justifying
\tableofcontents
\newpage

\pagestyle{plain}

\section{Planteamiento del Problema}
En la actualidad, el uso de los teléfonos inteligentes se ha consolidado como un elemento 
esencial en la vida cotidiana, al facilitar la comunicación, el acceso a la información, el entretenimiento y el desarrollo de actividades tanto acádemicas como laboral. No obstante, diversos estudios han evidenciado que el incremento sostenido en el tiempo de uso de smartphones ha dado lugar a hábitos tecnológicos poco saludables, especialmente en jóvenes adultos y estudiantes universitarios  \cite{ELHAI2017251}, \cite{MontagWalla2016}. Esta población presenta una alta exposición diaria a los dispositivos electrónicos, lo cual plantea desafíos significativos en términos de bienestar y autorregulación del uso de la tecnología.

La elevada disponibilidad y portabilidad del smartphone, junto con la necesidad constante de interacción digital, han favorecido el incremento progresivo del tiempo de uso del dispositivo entre los jóvenes adultos. Investigaciones recientes evidencian que esta población dedica, en promedio, entre 4 y 6 horas diarias al teléfono móvil, consolidando patrones de exposición prolongada a lo largo del día \cite{Andrews2015}. Este uso intensivo no solo incrementa la dependencia hacia el dispositivo, sino que también interfiere en la organización de las actividades cotidianas y en la capacidad de autorregulación del tiempo de uso, configurando un escenario propicio para la aparición de conductas problemáticas asociadas al teléfono inteligente. 

Adicionalmente, se ha identificado que el uso del smartphone durante horas previas al descanso es una práctica ampliamente extendida entre los jóvenes adultos. Estudios recientes reportan que una proporción significativa de estudiantes universitarios utiliza el teléfono móvil antes de dormir, lo cual se asocia con una disminución en la calidad y duración del sueño \cite{Joshi2021}. Investigaciones basadas en mediciones objetivas del uso del teléfono móvil han demostrado que los individuos con mayores niveles de uso presentan una reducción significativa en el tiempo total del sueño y una mayor probabilidad de experimentar problemas relacionados con el descanso nocturno \cite{Lemola2015}.

En este contexto emerge la nomofobia, definida como el miedo o ansiedad experimentados ante la imposibilidad de utilizar el teléfono móvil o de mantenerse conectado \cite{YildirimCorreia2015}. Estudios realizados mediante instrumentos válidos como la Nomophobia Questionnaire (NMP-Q) indican que entre el 30\% y el 50\% de los jóvenes adultos presentan niveles moderados o severos de nomofobia, lo que evidencia una alta prevalencia de este fenómeno en entornos universitarios \cite{YildirimCorreia2015},\cite{LeonMejia2021}. Estos resultados confirman que la nomofobia constituye una problemática relevante asociada a patrones de dependencia tecnológica. 

Diversas investigaciones han evidenciado que el uso excesivo del smartphone y la dependencia hacia este dispositivo se asocian con múltiples consecuencias negativas en la salud y el bienestar en los jóvenes adultos. En particular, se ha identificado una relación significativa entre el uso prolongado del teléfono móvil y el incremento de síntomas de ansiedad, estrés y dificultades en la regulación emocional \cite{ELHAI2020105962}, \cite{healthcare11142066}. Así mismo, investigaciones previas han demostrado que la utilización frecuente del smartphone, especialmente durante horas nocturnas, se asocia con alteraciones en la calidad y duración del sueño, lo que repercute negativamente en el descanso y en el funcionamiento diario de los estudiantes universitarios \cite{EXELMANS201693}. Estas afectaciones pueden derivar, a su vez, en una disminución del rendimiento académico, problemas de concentración y deterioro de las relaciones sociales, en la calidad de los jóvenes adultos \cite{SAMAHA2016321}, \cite{LEPP2014343}.

A pesar del creciente interés académico en el estudio de la nomofobia y del uso problemático del smartphone, la mayoría de las estrategias reportadas se centran en enfoques educativos o intervenciones psicológicas tradicionales. Si bien existen aplicaciones móviles orientadas a limitar el tiempo de uso del dispositivo, estas suelen basarse en mecanismos de control externo que pueden afectar la aceptación y adherencia por parte de los usuarios \cite{vanVelthoven2018}. En particular, se identifica una limitada disponibilidad de soluciones tecnológicas que integren estrategias de autorregulación apoyadas en retroalimentación háptica y sonora como mecanismo orientado a la reducción frente al uso problemático del smartphone \cite{Hampton2025}.

Dadas las circunstancias descritas, es posible observar que la nomofobia actúa como una telaraña invisible que atrapa progresivamente a los jóvenes adultos, condicionando sus hábitos cotidianos y su relación con el smartphone. En este escenario, se evidencia la necesidad de diseñar e implementar una aplicación móvil orientada a la reducción de la nomofobia, que aproveche las capacidades sensoriales del propio dispositivo para generar alertas no intrusivas y promover hábitos de uso más conscientes y equilibrados. La incorporación de retroalimentación háptica y sonora permite proponer una solución tecnológica innovadora, alineada con los principios del bienestar digital y con las tendencias actuales en el desarrollo de aplicaciones centradas en el usuario, contribuyendo a la mitigación de los efectos asociados al uso excesivo del teléfono inteligente.

En este contexto, surge la siguiente pregunta de investigación: 

\textit{¿Cómo puede una aplicación móvil con retroalimentación háptica y sonora contribuir a la reducción de los niveles de nomofobia, medidos mediante el Nomophobia Questionnaire (NMP-Q), en jóvenes adultos?}

\textbf{Hipótesis} 

Se plantea que el uso de una aplicación móvil con retroalimentación háptica y sonora contribuye significativamente a la reducción de los niveles de nomofobia, medidos mediante el Nomophobia Questionnaire (NMP-Q), en jóvenes de 18 a 26 años de edad, favoreciendo procesos de autorregulación del uso del smartphone y la adopción de hábitos tecnológicos más saludables.

Como hipótesis nula, se considera que la utilización de dicha aplicación móvil no produce cambios significativos en los niveles de nomofobia, medidos mediante el NMP-Q, en jóvenes adultos.


\section{Estado del Arte}
Este apartado presenta una revisión de estudios previos relacionados con la nomofobia, entendida como una manifestación del uso problemático del smartphone, los instrumentos utilizados para su medición y las consecuencias asociadas a este fenómeno en el bienestar y desempeño de los jóvenes adultos. Asimismo, se consideran trabajos que proponen soluciones tecnológicas orientadas a la regulación del uso del dispositivo móvil, así como estudios relacionados con el uso en retroalimentación sensorial en sistemas interactivos móviles.

A partir de esta revisión, se busca establecer un marco de referencia que permita contextualizar la propuesta del presente proyecto, evidenciando la necesidad de desarrollar una aplicación móvil que integre mecanismos de autorregulación apoyados en retroalimentación háptica y sonora como estrategia de reducción frente a la nomofobia.

\subsection{Uso del smartphone y aplicación de la nomofobia en jóvenes adultos}

El uso del teléfono inteligente se ha incrementado de manera acelerada en las últimas décadas, especialmente entre los adolescentes y los jóvenes adultos, convirtiéndose en una herramienta indispensable para la comunicación, el acceso a la información y la interacción social. Estudios a nivel global evidencian que la adopción del smartphone ha crecido con mayor rapidez en las generaciones jóvenes, consolidando patrones de uso intensivo y frecuente a lo largo del día \cite{Silver2019}, \cite{ijerph17020580}. En contextos universitarios, esta alta disponibilidad del dispositivo ha favorecido la aparición de hábitos tecnológicos prolongados que, en ciertos casos, derivan en conductas problemáticas asociadas al uso excesivo del teléfono móvil \cite{MontagWalla2016}, \cite{Ali31122024}.

Diversas investigaciones han señalado que el uso intensivo del smartphone se relaciona con dificultades en la autorregulación del tiempo, dependencia psicológica y una necesidad constante de conexión digital. Revisiones sistemáticas recientes destacan que los jóvenes universitarios constituyen una población particularmente vulnerable al desarrollo de comportamientos de uso problemáticos del smartphone, debido a factores como la presión académica, el uso de redes sociales y la integración del dispositivo móvil en actividades académicas y recreativas \cite{MontagWalla2016}, \cite{EXELMANS201693}, \cite{healthcare11142066}.

Desde una perspectiva conceptual, la nomofobia ha sido abordada en la literatura como un constructo emergente vinculado a la dependencia psicológica y conductual del smartphone. Los primeros estudios permitieron identificar este fenómeno como una respuesta de ansiedad asociada a la imposibilidad de acceder al dispositivo o mantenerse conectado, lo que motivó su análisis desde enfoques psicológicos y conceptuales \cite{YildirimCorreia2015}. Posteriormente, el desarrollo y validación de instrumentos psicométricos, como el Nomophobia Questionnaire (NMP-Q), permitió operacionalizar el concepto y analizar sus dimensiones subyacentes, facilitando estudios comparativos en distintos contextos culturales y poblacionales \cite{ijerph17020580}.

A partir de la aplicación del NMP-Q, diversos estudios han confirmado la presencia significativa de la nomofobia en jóvenes adultos y estudiantes universitarios, reportando prevalencias elevadas de niveles moderados y severos en distintos contextos académicos \cite{Ali31122024}, \cite{Aslani2025}. Más allá de su magnitud, la literatura coincide en que este fenómeno se asocia de manera consistente con consecuencias negativas en el bienestar y el desempeño de esta población. Investigaciones empíricas y revisiones sistemáticas han evidenciado relaciones significativas entre la nomofobia y el incremento de síntomas de ansiedad, estrés y alteraciones emocionales, así como con problemas en la calidad de sueño y el rendimiento académico, especialmente en individuos con patrones de uso intensivo del smartphone \cite{EXELMANS201693}, \cite{healthcare11142066}, \cite{Cha2023}. Estos hallazgos consolidan la nomofobia como una manifestación relevante del uso problemático del teléfono inteligente, con implicaciones directas en la salud y la calidad de vida de los jóvenes adultos.

Desde una perspectiva institucional, organismos internacionales como la Organización Mundial de la Salud (OMS) han advertido sobre el impacto del uso excesivo de pantallas en la salud mental de adolescentes y jóvenes, destacando la necesidad de estrategias de autorregulación orientadas a la promoción de hábitos digitales saludables \cite{WHO2024TeensScreens}. Estas recomendaciones refuerzan la importancia de abordar la nomofobia no solo como un fenómeno descriptivo, sino como una problemática susceptible a mecanismos de autorregulación del uso del dispositivo.

En conjunto, la literatura revisada evidencia que la nomofobia constituye una problemática creciente en jóvenes adultos, estrechamente vinculada a patrones de uso intensivo del smartphone y a diversas consecuencias negativas en la salud y la calidad de vida. Si bien los estudios analizados han permitido caracterizar el fenómeno, identificar sus principales consecuencias y desarrollar instrumentos de medición validados, resulta necesario profundizar en el análisis de las estrategias de regulación propuestas en la literatura, particularmente aquellas basadas en soluciones tecnológicas orientadas a la autorregulación del uso del smartphone. Este análisis se aborda en los apartados siguientes del estado del arte.

\subsection{Trabajos Relacionados}

Con el objetivo de analizar las principales propuestas de intervención asociadas a la nomofobia y el uso problemático del smartphone, se realizó una revisión dirigida de trabajos relacionados reportados en la literatura científica. Esta revisión permitió identificar enfoques educativos, psicológicos, y tecnológicos previamente aplicados, así como sus principales alcances y limitaciones.

La búsqueda de información se llevó a cabo en base de datos y repositorios académicos reconocidos, tales como ScienceDirect, PubMed, MDPI, Elsevier entre otros y repositorios universitarios, empleando palabras clave como “nomophobia”, “problematic smartphone use”, “digital intervention”, “mobile application”, “self-regulation” y “Nomophobia Questionnaire (NMP-Q)”. Los artículos fueron seleccionados considerando su pertinencia temática, población de estudio y aporte al análisis de estrategias de intervención, priorizando estudios revisados por pares y propuestas con validación empírica.

A continuación, se describen los trabajos más relevantes identificados, destacando sus metodologías, resultados principales y contribuciones al abordaje de la nomofobia y la regulación del uso del smartphone.

\subsubsection{Intervenciones psicoeducativas y conductuales}

Algunos de los primeros enfoques para abordar la nomofobia se han centrado en intervenciones psicoeducativas presenciales, orientadas a fortalecer la conciencia sobre el uso del smartphone y a promover habilidades de autorregulación. En este contexto, el estudio Family Supported Nomophobia Reduction Intervention for Adolescents \cite{CelikKucukKulaberoglu2024} evaluó la eficacia de un programa estructurado que involucró tanto a adolescentes como a sus familias. Los resultados mostraron una reducción significativa y sostenida de la nomofobia en el grupo intervenido, evidenciando el impacto positivo del acompañamiento familiar y la educación digital. Sin embargo, la intervención se limita a un entorno presencial y a un periodo de tiempo específico, lo que dificulta su escalabilidad y continuidad.

De manera similar, The Role of Mindfulness and Digital Detox to Adolescent Nomophobia \cite{AiniBukhoriBakar2021} analizó el efecto de sesiones de mindfulness y estrategias de desconexión digital en adolescentes, reportando disminuciones significativas en los niveles de nomofobia y ansiedad. Aunque estos resultados confirman la efectividad de las estrategias de concienciación y control voluntario del uso del dispositivo, el estudio evidencia la necesidad de herramientas tecnológicas que permitan mantener estos hábitos de forma constante fuera del entorno terapéutico.

\subsubsection{Aplicaciones móviles orientadas a autorregulación y cambio de hábitos}

Con el avance de la tecnología móvil, diversas investigaciones han explorado el uso de aplicaciones móviles como herramientas de intervención digital. En esta línea, A Mobile Intervention for Self-Efficacious and Goal-Directed Smartphone Use \cite{Keller2021} evaluó la aplicación Not Less But Better, diseñada para fomentar un uso más consciente del smartphone mediante planificación, establecimiento de metas y fortalecimiento de la autoeficacia. Los resultados del ensayo controlado indicaron una reducción del uso problemático del dispositivo, aunque la aplicación no está orientada específicamente a la nomofobia ni incorpora estímulos sensoriales como retroalimentación háptica o sonora.

Desde una perspectiva más general, el trabajo Mobile health in primary care \cite{ALOS2024102900} analiza el potencial de las soluciones de mHealth para promover el bienestar, destacando su capacidad de personalizar intervenciones y empoderar a los usuarios. No obstante, los autores señalan desafíos importantes relacionados con la validación clínica, la usabilidad y la sostenibilidad de estas aplicaciones, aspectos críticos para el diseño de herramientas orientadas a la reducción del uso problemático del smartphone.

\subsubsection{Aplicaciones de medición y detección temprana}

Algunos trabajos han priorizado la medición y detección de la nomofobia, más que su intervención directa. En este sentido, Girassol: A Mobile App to Measure Levels of Nomophobia in Adolescents and Young People \cite{CunhaSouzaSantiago2020} presenta una aplicación diseñada para evaluar niveles de dependencia tecnológica mediante cuestionarios validados. La herramienta mostró altos niveles de aceptación y utilidad, permitiendo identificar patrones de riesgo; sin embargo, su alcance se limita al diagnóstico y no incorpora mecanismos de regulación.

\subsubsection{Intervenciones digitales basadas en persuasión y automonitoreo}

Otras investigaciones se han enfocado en el auto-monitoreo y la persuasión digital como estrategias para modificar hábitos de uso. NUGU \cite{KoYangLee2015} propone una aplicación grupal que utiliza el apoyo social y la responsabilidad compartida para fomentar la autorregulación del uso del smartphone, demostrando que la dinámica colectiva puede potenciar el cambio de conducta. 

En esta misma línea, MyTime \cite{Hiniker2016} permitió a los usuarios establecer metas personales y limitar aplicaciones consideradas improductivas, logrando reducciones significativas en el uso no deseado sin afectar actividades valoradas. Complementariamente, TILT \cite{FoulonneauCalvaryVillain2016} emplea mensajes persuasivos adaptativos basados en el contexto y el historial de uso, mostrando reducciones sostenidas en el tiempo de uso del smartphone. Por su parte, AppDetox \cite{LoechtefeldBoehmerGanev2013} se centra en la creación de reglas autoimpuestas para restringir aplicaciones específicas, evidenciando que los usuarios tienden a limitar principalmente redes sociales y mensajería.

Aunque estas aplicaciones demuestran la efectividad del auto-monitoreo y la persuasión, su enfoque se orienta principalmente al control conductual del tiempo o del acceso, sin abordar de manera explícita la ansiedad asociada a la desconexión ni integrar retroalimentación sensorial como apoyo al proceso de autorregulación.

\subsubsection{Aplicaciones orientadas al contexto educativo}

Finalmente, Flipd App Reduces Cellular Phone Distractions in the Traditional College Classroom \cite{Neuwirth2020} evaluó el uso de una aplicación para reducir distracciones digitales en clases universitarias. Los resultados evidenciaron mejoras en la asistencia y participación estudiantil; sin embargo, el enfoque de Flipd se limita al entorno académico y al bloqueo temporal del dispositivo, sin considerar la dimensión emocional del uso problemático del smartphone.

A continuación, se presenta un resumen de los enfoques abordados en los trabajos encontrados en la literatura:


{
    \footnotesize
    \renewcommand{\arraystretch}{1.1}

    \begin{longtable}{c p{3cm} p{3cm} p{3cm} p{3cm}}
        
        \caption{\centering{Resumen de Trabajos Relacionados.}}
        \label{tab::estadoArteNomofobia} \\

        \hline
        
        \textbf{Ref} & \textbf{Artículo} & \textbf{Tipo de Intervención y Enfoque} & \textbf{Principales Aportes} & \textbf{Limitaciones Identificadas} \\ \hline
        \endfirsthead

        \cite{CelikKucukKulaberoglu2024} & Family Supported Nomophobia Reduction Intervention for Adolescents & Intervención psicoeducativa presencial con enfoque en educación digital, autorregulación y acompañamiento familiar & Reducción significativa y sostenida de la nomofobia; evidencia el impacto positivo del entorno familiar & Limitada al contexto presencial; baja escalabilidad y ausencia de soporte tecnológico continuo \\ 

        \hline

        \cite{AiniBukhoriBakar2021} & The Role of Mindfulness and Digital Detox to Adolescent Nomophobia & Intervención conductual presencial basada en mindfulness y desconexión digital voluntaria & Disminución de nomofobia y ansiedad; refuerza la conciencia sobre el uso del smartphone & No integra herramientas tecnológicas ni seguimiento fuera del entorno terapéutico \\

        \hline

        \cite{Keller2021} & Not Less But Better & Aplicación móvil orientada a autorregulación, planificación del uso y fortalecimiento de la autoeficacia &Reducción del uso problemático del smartphone mediante metas y reflexión & No está dirigida específicamente a la nomofobia ni incorpora retroalimentación sensorial \\

        \hline

        \cite{ALOS2024102900} & Mobile health in primary care & Soluciones de mHealth orientadas a promoción del bienestar y personalización de intervenciones & Resalta el potencial del mHealth para empoderar usuarios y promover hábitos saludables & Señala desafíos de validación clínica, usabilidad y sostenibilidad \\

        \hline

        \cite{CunhaSouzaSantiago2020} & Girassol & Aplicación móvil para medición y detección temprana de nomofobia mediante cuestionarios validados & Alta aceptación y utilidad para identificar patrones de riesgo & Enfoque limitado al diagnóstico; no incluye intervención activa \\

        \hline

        \cite{KoYangLee2015} & NUGU & Aplicación móvil grupal basada en apoyo social y autorregulación colectiva &Evidencia que la dinámica social favorece la reducción del uso excesivo del smartphone & No aborda explícitamente la nomofobia ni el componente emocional \\

        \hline

        \cite{Hiniker2016} & MyTime & Aplicación de autorregulación basada en metas personales y control voluntario del uso & Reducción del uso improductivo sin afectar actividades valoradas & No considera ansiedad por desconexión ni retroalimentación sensorial \\

        \hline

        \cite{FoulonneauCalvaryVillain2016} & TILT & Aplicación persuasiva con auto-monitoreo y mensajes adaptativos según contexto & Reducción sostenida del tiempo de uso mediante persuasión contextual & Enfoque centrado en control conductual, no en nomofobia \\

        \hline

        \cite{LoechtefeldBoehmerGanev2013} & AppDetox & Aplicación de autorregulación basada en reglas autoimpuestas y restricción de apps & Evidencia efectividad del autocontrol, especialmente sobre redes sociales & Basada en bloqueo; no aborda ansiedad ni procesos emocionales \\

        \hline

        \cite{Neuwirth2020} & Flipd & Aplicación educativa para limitar el uso del smartphone en el aula mediante bloqueo temporal & Mejora asistencia y participación académica & Aplicación restringida al contexto educativo; ignora la dimensión emocional \\

        \hline
    \end{longtable}
}

En síntesis, la Tabla \ref{tab::estadoArteNomofobia} resume los principales trabajos relacionados que abordan la nomofobia y el uso problemático del smartphone, mediante intervenciones psicoeducativas, aplicaciones móviles y estrategias de autorregulación. La revisión evidencia que la mayoría de las propuestas se enfocan en el control del tiempo de uso o en el auto monitoreo del dispositivo, con un alcance limitado en el abordaje de la ansiedad asociada a la desconexión y sin integrar mecanismos de retroalimentación sensorial. Estos aspectos permiten identificar oportunidades de análisis que se profundizan en el apartado siguiente.

\subsection{Conclusiones}

\begin{itemize}
    \item La nomofobia constituye una problemática creciente en jóvenes adultos, asociada de forma consistente al uso intensivo del smartphone y a consecuencias negativas en la salud mental, la calidad del sueño y el rendimiento académico, lo que evidencia la necesidad de estrategias orientadas a la regulación de su uso.
    \item Las soluciones tecnológicas existentes para abordar el uso problemático del smartphone presentan limitaciones relevantes, ya que se enfocan principalmente en el control del tiempo de pantalla y el bloqueo de aplicaciones, sin evaluar de manera específica los niveles de nomofobia ni abordar la dependencia psicológica asociada al miedo a la desconexión.
    \item En este contexto, el estado del arte revela un vacío en el desarrollo de aplicaciones móviles que integren retroalimentación háptica y sonora como mecanismo central orientado a la reducción del uso problemático del smartphone, a pesar de su potencial para promover la conciencia del uso y fortalecer la autorregulación de manera intuitiva y no intrusiva, lo que justifica la propuesta del presente proyecto.
\end{itemize}

\subsection{Aportes al proyecto}

El estado del arte revela un vacío en el desarrollo de aplicaciones móviles que integren retroalimentación háptica y sonora como mecanismo central de intervención no intrusiva, orientado a la reducción del uso problemático del smartphone, a pesar de su potencial para promover la conciencia del uso y fortalecer la autorregulación, lo que justifica la propuesta del presente proyecto.

Como aporte innovador, se propone la implementación de mecanismos de retroalimentación háptica y sonora mediante vibraciones controladas y señales sonoras breves, concebidos como estrategias de autorregulación personalizadas que facilitan la toma de conciencia sobre patrones de uso de riesgo, sin recurrir a mecanismos restrictivos invasivos.

Adicionalmente, el proyecto se enfoca en jóvenes adultos entre 18 y 26 años, una población identificada en la literatura como altamente vulnerable a la nomofobia, contribuyendo al abordaje del problema mediante una intervención tecnológica orientada a la reducción de sus niveles desde una perspectiva de salud mental. Finalmente, el trabajo fortalece el ámbito académico al servir como base para futuras investigaciones en bienestar digital y aplicaciones móviles orientadas a la regulación del uso del smartphone.

\section{Objetivos}
\subsection{Objetivo General}

Diseñar e implementar una aplicación móvil con retroalimentación háptica y sonora orientada a la prevención de la nomofobia en jóvenes adultos, promoviendo un uso más consciente y equilibrado del smartphone mediante estrategias de monitoreo y autorregulación del uso del dispositivo.

\subsection{Objetivos Especificos}

\begin{itemize}
    \item Caracterizar los patrones de uso del smartphone en jóvenes adultos entre 18 y 26 años, evaluando sus niveles de nomofobia mediante el cuestionario validado NMP-Q, con el fin de identificar hábitos de uso y los factores de riesgo asociados a esta dependencia tecnológica.
    \item Diseñar e implementar una aplicación móvil que integre el monitoreo del uso del smartphone y mecanismos de retroalimentación háptica y sonora, orientados a fomentar la autorregulación y el uso consciente del dispositivo.
    \item Evaluar la efectividad de la aplicación móvil mediante pruebas con usuarios, comparando los niveles de nomofobia antes y después de la intervención a través del cuestionario NMP-Q.
\end{itemize}

\section{Metodología}
Dado que el objetivo del proyecto es diseñar e implementar una aplicación móvil orientada a control de la nomofobia en jóvenes, la metodología seleccionada es el \textit{Diseño Centrado en el Usuario (DCU)}, ya que permite comprender las necesidades reales de los jóvenes, definir requisitos funcionales y de experiencia de uso, y evaluar la aplicación mediante pruebas y medición de niveles de nomofobia antes y después de la intervención. 
El enfoque metodológico se estructura en fases secuenciales e iterativas, garantizando que la solución propuesta responda de manera efectiva a los patrones de uso del smartphone y favorezca la autorregulación del comportamiento digital.

\begin{enumerate}
    \item \textbf{Análisis del usuario y contexto} \\
    En esta fase se realiza la caracterización de los jóvenes participantes, identificando su nivel inicial de nomofobia mediante la aplicación del cuestionario validado NMP-Q. Este análisis permite comprender los patrones de uso del smartphone, identificar necesidades reales y definir los escenarios de uso de la aplicación móvil. \\

    \item \textbf{Definición de requisitos funcionales y no funcionales} \\
    Con base en la información obtenida durante el análisis del usuario y la revisión bibliográfica, se establecen los requisitos funcionales y no funcionales de la aplicación. Estos incluyen funcionalidades como el monitoreo del uso del smartphone, la aplicación del cuestionario NMP-Q, la visualización de datos y la implementación de mecanismos de retroalimentación háptica y sonora orientados a la autorregulación del usuario. \\

    \item \textbf{Diseño de prototipos y experiencia de usuario} \\
    En esta etapa se desarrollan prototipos iniciales que permiten estructurar la interfaz de la aplicación y definir las principales interacciones con el usuario. Los prototipos son evaluados mediante pruebas preliminares con usuarios, lo que permite identificar oportunidades de mejora y ajustar el diseño antes de la implementación definitiva, garantizando una experiencia de uso intuitiva y comprensible. \\

    \item \textbf{Desarrollo iterativo de la aplicación} \\
    A partir de los prototipos validados, se lleva a cabo el desarrollo de la aplicación móvil mediante ciclos iterativos de diseño, prueba y ajuste. Para este proceso se adopta un enfoque ágil basado en Scrum, lo que facilita la planificación, el control del progreso y la incorporación continua de retroalimentación por parte de los usuarios. \\

    \item  \textbf{Implementación} \\
    En esta fase se realiza el desarrollo completo de la aplicación móvil, integrando las funcionalidades definidas previamente, tales como el monitoreo del uso del smartphone, la aplicación del cuestionario NMP-Q y los mecanismos de retroalimentación háptica y sonora. El desarrollo se realiza de manera iterativa, incorporando pruebas internas y revisiones periódicas para garantizar la estabilidad, el correcto funcionamiento y la coherencia con los requisitos establecidos. \\

    \item \textbf{Pruebas con usuarios y evaluación de usabilidad} \\
    Una vez implementado el prototipo funcional, se llevan a cabo pruebas con usuarios durante un periodo de uso definido. En esta etapa se evalúan aspectos como la facilidad de uso, la comprensión de la interfaz, la eficiencia en la interacción y el nivel de satisfacción del usuario. Asimismo, se recopilan comentarios cualitativos que permiten realizar ajustes antes de la versión final de la aplicación. \\

    \item \textbf{Medición del impacto (NMP-Q antes y después)} \\
    Para evaluar el impacto de la aplicación sobre la nomofobia, se aplica el cuestionario NMP-Q en dos momentos:
    \begin{itemize}
        \item \textit{Fase inicial (pre-test):} antes del uso de la aplicación.
        \item \textit{Fase final (post-test):} después de un periodo de uso definido.
    \end{itemize}
    Los resultados obtenidos permiten analizar los cambios en los niveles de nomofobia y en la adopción de hábitos de autorregulación, mediante la comparación de las mediciones iniciales y finales, con el fin de determinar el efecto de la aplicación móvil en los usuarios. \\

    \item \textbf{Documentación y sistematización} \\
    Se realiza la documentación detallada de todas las fases del proyecto, incluyendo las decisiones de diseño, los ajustes realizados y el análisis de los resultados obtenidos a partir del cuestionario NMP-Q. Esta documentación sirve como base para la redacción del proyecto de grado. \\

    \item \textbf{Sustentación} \\
    Finalmente, se presenta el proyecto de grado mediante la sustentación de los resultados obtenidos, las conclusiones alcanzadas y las recomendaciones para futuras mejoras o investigaciones relacionadas.
\end{enumerate}


\section{Actividades y Cronograma}
\subsection{Actividades}
El proyecto se desarrollará durante un periodo de nueve meses, siguiendo la metodología de Diseño Centrado en el Usuario, estructurada en siete fases secuenciales e iterativas, cada una con un propósito específico.

\textbf{Actividad 1. Análisis del usuario y del contexto}

Se realiza una revisión bibliográfica y un diagnóstico inicial mediante la aplicación del cuestionario NMP-Q, con el fin de identificar los niveles de nomofobia y los patrones de uso del smartphone en los participantes. Esta actividad permite comprender las necesidades reales de los usuarios.

\textbf{Actividad 2. Definición de requisitos}

Con base en la información recolectada, se establecen los requisitos funcionales y no funcionales de la aplicación, definiendo las características técnicas y de experiencia de usuario que debe cumplir la solución propuesta.

\textbf{Actividad 3. Diseño de la solución}

Se diseñan las interfaces y se elaboran los prototipos de la aplicación móvil, priorizando la usabilidad, la claridad y la eficiencia. El diseño busca facilitar una interacción intuitiva que favorezca el uso consciente y autorregulado del smartphone.

\textbf{Actividad 4. Desarrollo de la aplicación:}

Se implementan las funcionalidades principales de la aplicación, incluyendo el cuestionario NMP-Q, el monitoreo del uso del smartphone y los mecanismos de retroalimentación háptica y sonora. Asimismo, se realizan pruebas funcionales para garantizar su correcto desempeño.

\textbf{Actividad 5. Evaluación con usuarios:}

Se lleva a cabo la evaluación del sistema mediante pruebas de usabilidad durante un periodo de uso controlado, analizando aspectos como la facilidad de uso, la comprensión de las funciones, la eficiencia en la realización de tareas y el nivel de satisfacción del usuario.

\textbf{Actividad 6. Ajustes y optimización:}

Con base en los resultados de la evaluación, se realizan ajustes en la interfaz y en las funcionalidades de la aplicación para mejorar su desempeño y experiencia de uso.

\textbf{Actividad 7. Documentación y cierre:}

Se documenta todo el proceso de desarrollo, se analizan los resultados finales y se elabora el informe del proyecto de grado para su respectiva sustentación.

\subsection{Cronograma de Actividades}

\begin{figure}[H]
    \centering
    \includegraphics[width=0.95\textwidth]{figuras/CronogramaActividades.png}
    \caption{Cronograma de Actividades.}
    \label{fig::CronogramaActividades}
\end{figure}


\newpage

\section{Recursos, Presupuesto y Fuentes de Financiación}
Se presenta el presupuesto del proyecto distribuido por categorías y fuentes de financiación. La Tabla \ref{tab:presupuesto} fue elaborada siguiendo los lineamientos establecidos por la Facultad de Ingeniería Electrónica y Telecomunicaciones de la Universidad del Cauca para la formulación de presupuestos en anteproyectos \cite{UnicaucaGuiaAnteproyecto}, considerando un valor de \$22.358 COP por punto para el año 2025. El presupuesto corresponde a una duración total del proyecto de nueve (9) meses, equivalentes a treinta y seis (36) semanas de trabajo.

Para la estimación de los costos se tuvieron en cuenta los siguientes criterios:

\begin{itemize}
    \item \textbf{Recursos Humanos:} \\
    La dirección del proyecto estará a cargo de un director y un codirector, a quienes se les asigna una dedicación de 2 horas semanales, con un valor de 2.5 puntos por hora, durante un periodo de 36 semanas, conforme a la normativa institucional.

    La ejecución del proyecto estará a cargo de dos estudiantes, a quienes se les asigna una dedicación de 30 horas semanales por estudiante, con un valor de 1.5 puntos por hora, durante 36 semanas. Esta dedicación corresponde al aporte académico requerido para el desarrollo del trabajo de grado. \\

    \item \textbf{Recursos Técnicos:} \\
    Se considera el uso de equipos de cómputo personales, cuyo costo se estima como el 33\% del valor comercial actual, así como el uso de dispositivos móviles inteligentes con sistema operativo Android, necesarios para el desarrollo, implementación y validación de la aplicación móvil propuesta. \\

    \item \textbf{Recursos Software:} \\
    Se contempla el uso de servicios de bases de datos en la nube, específicamente Google Cloud Firestore, considerando un 20\% del costo estimado de adquisición, correspondiente a un escenario de uso parcial durante las fases de desarrollo y prueba del sistema. Asimismo, se incluyen herramientas de software para el desarrollo y diseño de la aplicación. \\

    \item \textbf{Recursos Bibliográficos:} \\
    Se estima un costo asociado al acceso a documentación científica especializada, correspondiente a la consulta de artículos y publicaciones académicas no cubiertas por las licencias institucionales. \\

    \item \textbf{A.U.I (Administración, Utilidades e Imprevistos):} \\
    El rubro de Administración, Utilidad e Imprevistos (A.U.I.) corresponde al 20\% del costo total del proyecto, e incluye los gastos relacionados con infraestructura física, gestión administrativa y una reserva para posibles imprevistos durante el desarrollo del proyecto.
\end{itemize}

\begin{table}[ht]
\centering
\caption{Presupuesto del proyecto}
\label{tab:presupuesto}
\renewcommand{\arraystretch}{1.25}
\begin{tabular}{p{4.2cm}ccc}
\toprule
\multirow{2}{*}{\textbf{Rubros}} & \multicolumn{2}{c}{\textbf{Fuentes}} & \multirow{2}{*}{\textbf{Total}} \\
\cmidrule(lr){2-3}
 & \textbf{Estudiantes} & \textbf{Departamento} &  \\
\midrule

\multicolumn{4}{l}{\textbf{Recursos Humanos}} \\
Director & -- & \$4'024.440 & \$4'024.440 \\
Codirector & -- & \$4'024.440 & \$4'024.440 \\
Estudiantes (2) & \$72'439.920 & -- & \$72'439.920 \\
\addlinespace

\multicolumn{4}{l}{\textbf{Recursos Técnicos}} \\
\textit{Recursos Hardware} \\
Utilización de PC & \$1.650.000 & -- & \$1.650.000 \\
Utilización de Smartphone & \$660.000 & -- & \$660.000 \\
Otros & \$300.000 & -- & \$300.000 \\
\addlinespace

\textit{Recursos Software} \\
Servicios de Google Cloud Firestore & \$360.000 & -- & \$360.000 \\
Herramientas Software & \$150.000 & -- & \$150.000 \\
\addlinespace

\multicolumn{4}{l}{\textbf{Recursos Bibliográficos}} \\
Documentación & \$250.000 & -- & \$250.000 \\
\midrule

Sub Total & \$75'809.920 & \$8'048.880 & \$83'858.800 \\
A.U.I (20\%) & \$15'161.984 & \$1'609.776 & \$16'771.760 \\
\midrule
\textbf{Total} & \textbf{\$90'971.904} & \textbf{\$9'658.656} & \textbf{\$100'630.560} \\
\bottomrule
\end{tabular}
\end{table}


\newpage

\section{Condiciones de Entrega}
Al término de este proyecto de investigación se entregarán los siguientes productos:

\begin{itemize}
    \item \textbf{Un documento principal:} que incluirá el estado del arte, los conceptos teóricos, la revisión de soluciones tecnológicas y los resultados experimentales obtenidos. \\
    \item \textbf{Material documental adicional:} elaborado para complementar y ampliar el trabajo de grado. \\
    \item \textbf{Un repositorio en GitHub:} en el cual se almacenará la información generada durante la investigación, incluyendo el código fuente de la solución, archivos ejecutables, documentación técnica y demás recursos relevantes. \\
    \item \textbf{Un artículo académico:} en el que se presentará el desarrollo del proyecto de tesis junto con la solución propuesta.
\end{itemize}

\newpage

% ------------------------------------------ %
% Bibliografía (BibTeX - IEEE)
% ------------------------------------------ %
\bibliographystyle{IEEEtran}
\bibliography{Bibliografia}

\newpage
\section*{}
\addcontentsline{toc}{section}{Acta de Propiedad Intelectual}
\begin{center}
\bfseries UNIVERSIDAD DEL CAUCA \\ 
\vspace{0.3cm}
 FACULTAD DE INGNERÍA ELECTRÓNICA Y TELECOMUNICACIONES \\
 \vspace{0.3cm} 
 ACTA DE ACUERDO SOBRE LA PROPIEDAD INTECLECTUAL DEL TRABAJO DE GRADO
\end{center}

En atención al acuerdo del Honorable Consejo Superior de la Universidad del Cauca, número 008 del 23 de febrero de 1999, donde se estipula todo lo concerniente a la producción intelectual en la institución, los abajo firmantes, reunidos el día del mes de  Enero del año 2026 en el salón del Consejo de Facultad, acordamos las siguientes condiciones para el desarrollo y posible usufructo del siguiente trabajo:

\textbf{Material del Acuerdo:} Trabajo de grado para optar al título de Ingeniero en Electrónica y Telecomunicaciones.

\textbf{Título del Proyecto:} Diseño e implementación de una aplicación móvil con retroalimentación háptica y sonora para la reducción de la nomofobia en jóvenes adultos.

\textbf{Objetivo del Proyecto:} Diseñar e implementar una aplicación móvil con retroalimentación háptica y sonora orientada a la reducción de los niveles de nomofobia en jóvenes adultos, promoviendo un uso más consciente y equilibrado del smartphone.

\textbf{Duración del Proyecto:} 9 meses.

Los participantes del proyecto, los señores estudiantes de pregrado MARIA JOSÉ CABRERA PANTOJA y FERNANDO MOLINA PLAZA, identificados con las cédulas de ciudadanía número 10.007.522.207 de Popayán y 1.152.714.291 de Medellín respectivamente, a quienes en adelante se les llamará “ESTUDIANTES”, el ingeniero CÉSAR ALBERTO COLLAZOS ORDÓÑEZ en calidad de Director del trabajo de grado, identificado con la cédula de ciudadanía número 76.309.486, y el ingeniero ŁUKASZ TOMCZYK en calidad de Codirector del trabajo de grado, identificado con el pasaporte EN3829009 y lugar Polonia, a quienes en adelante se les llamará “DOCENTES”, y la Universidad del Cauca, representada por el Decano de la FIET, manifiestan que:

\begin{enumerate}
    \item La idea original del proyecto es de Maria José Cabrera Pantoja y Fernando Molina Plaza, quien la propuso y presentó al Departamento de Sistemas, que la aceptó como tema para el proyecto de grado en referencia. \\
    \item La idea mencionada fue acogida por los estudiantes como proyecto para obtener el grado de Ingeniero en Electrónica y Telecomunicaciones, quienes la desarrollarán bajo la dirección del docente. \\
    \item Los derechos intelectuales y morales corresponden al docente y a los estudiantes. \\
    \item Los derechos patrimoniales corresponden al docente, a los estudiantes y a la Universidad del Cauca por partes iguales y continuarán vigentes, aún después de la desvinculación de alguna de las partes de la Universidad. \\
    \item Los participantes se comprometen a cumplir con todas las condiciones de tiempo, recursos, infraestructura, dirección y asesoría establecidas en el anteproyecto, a estudiar, analizar, documentar y hacer acta de cambios aprobados por el Consejo de Facultad durante el desarrollo del proyecto, los cuales entran a formar parte de las condiciones generales. \\
    \item Los estudiantes se comprometen a restituir en efectivo y de manera inmediata a la Universidad los aportes recibidos y los pagos hechos por la Institución a terceros por servicios o equipos, si el Comité de Investigaciones declara suspendido el proyecto por incumplimiento del cronograma o de las demás obligaciones contraídas por los estudiantes; y en cualquier caso de suspensión, la obligación de devolver en el estado en que les fueron proporcionados y de manera inmediata los equipos de laboratorio, de cómputo y demás bienes suministrados por la Universidad para la realización del proyecto. \\
    \item El docente y los estudiantes se comprometen a dar crédito a la Universidad y hacer mención del Fondo de Fomento de Investigación en los informes de avance y de resultados, y en registro de estos, cuando ha habido financiación de la Universidad o del Fondo. \\
    \item Cuando por razones de incumplimiento, legalmente comprobadas, de las condiciones de desarrollo planteadas en el anteproyecto y sus modificaciones, alguno de los participantes deba ser excluido del proyecto, los derechos aquí establecidos concluyen para él. Además, se tendrán en cuenta los principios establecidos en el reglamento estudiantil vigente de la Universidad del Cauca en lo concerniente a la cancelación y la pérdida del derecho a continuar estudios. \\
    \item El documento del anteproyecto y las actas de modificaciones, si las hubiere, forman parte integral de la presente acta. \\
    \item Los aspectos no contemplados en la presente acta serán definidos en los términos del acuerdo 008 del 23 de febrero de 1999 expedido por el Consejo Superior de la Universidad del Cauca, del cual los participantes aseguran tener pleno conocimiento. \\
\end{enumerate}
    
\vspace{1.5cm}

\begin{center}
    \begin{tabular}{c c}
        \rule{6.5cm}{0.4pt} & \rule{6.5cm}{0.4pt} \\
        César Alberto Collazos Ordóñez & Łukasz Tomczyk \\
        \textbf{Director} & \textbf{Codirector} \\
    \end{tabular}
\end{center}

\vspace{1cm}

\begin{center}
    \begin{tabular}{c c}
        \rule{6.5cm}{0.4pt} & \rule{6.5cm}{0.4pt} \\
        Maria José Cabrera Pantoja & Fernando Molina Plaza \\
        \textbf{Estudiante} & \textbf{Estudiante} \\
    \end{tabular}
\end{center}

\vspace{1cm}

\begin{center}
    \rule{6.5cm}{0.4pt} \\
    Alejandro Toledo Tovar \\
    \textbf{Decano FIET}
\end{center}





\end{document}