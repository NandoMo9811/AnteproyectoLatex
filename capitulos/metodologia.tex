Dado que el objetivo del proyecto es diseñar e implementar una aplicación móvil orientada a control de la nomofobia en jóvenes, la metodología seleccionada es el \textit{Diseño Centrado en el Usuario (DCU)}, ya que permite comprender las necesidades reales de los jóvenes, definir requisitos funcionales y de experiencia de uso, y evaluar la aplicación mediante pruebas y medición de niveles de nomofobia antes y después de la intervención. 
El enfoque metodológico se estructura en fases secuenciales e iterativas, garantizando que la solución propuesta responda de manera efectiva a los patrones de uso del smartphone y favorezca la autorregulación del comportamiento digital.

\begin{enumerate}
    \item \textbf{Análisis del usuario y contexto} \\
    En esta fase se realiza la caracterización de los jóvenes participantes, identificando su nivel inicial de nomofobia mediante la aplicación del cuestionario validado NMP-Q. Este análisis permite comprender los patrones de uso del smartphone, identificar necesidades reales y definir los escenarios de uso de la aplicación móvil. \\

    \item \textbf{Definición de requisitos funcionales y no funcionales} \\
    Con base en la información obtenida durante el análisis del usuario y la revisión bibliográfica, se establecen los requisitos funcionales y no funcionales de la aplicación. Estos incluyen funcionalidades como el monitoreo del uso del smartphone, la aplicación del cuestionario NMP-Q, la visualización de datos y la implementación de mecanismos de retroalimentación háptica y sonora orientados a la autorregulación del usuario. \\

    \item \textbf{Diseño de prototipos y experiencia de usuario} \\
    En esta etapa se desarrollan prototipos iniciales que permiten estructurar la interfaz de la aplicación y definir las principales interacciones con el usuario. Los prototipos son evaluados mediante pruebas preliminares con usuarios, lo que permite identificar oportunidades de mejora y ajustar el diseño antes de la implementación definitiva, garantizando una experiencia de uso intuitiva y comprensible. \\

    \item \textbf{Desarrollo iterativo de la aplicación} \\
    A partir de los prototipos validados, se lleva a cabo el desarrollo de la aplicación móvil mediante ciclos iterativos de diseño, prueba y ajuste. Para este proceso se adopta un enfoque ágil basado en Scrum, lo que facilita la planificación, el control del progreso y la incorporación continua de retroalimentación por parte de los usuarios. \\

    \item  \textbf{Implementación} \\
    En esta fase se realiza el desarrollo completo de la aplicación móvil, integrando las funcionalidades definidas previamente, tales como el monitoreo del uso del smartphone, la aplicación del cuestionario NMP-Q y los mecanismos de retroalimentación háptica y sonora. El desarrollo se realiza de manera iterativa, incorporando pruebas internas y revisiones periódicas para garantizar la estabilidad, el correcto funcionamiento y la coherencia con los requisitos establecidos. \\

    \item \textbf{Pruebas con usuarios y evaluación de usabilidad} \\
    Una vez implementado el prototipo funcional, se llevan a cabo pruebas con usuarios durante un periodo de uso definido. En esta etapa se evalúan aspectos como la facilidad de uso, la comprensión de la interfaz, la eficiencia en la interacción y el nivel de satisfacción del usuario. Asimismo, se recopilan comentarios cualitativos que permiten realizar ajustes antes de la versión final de la aplicación. \\

    \item \textbf{Medición del impacto (NMP-Q antes y después)} \\
    Para evaluar el impacto de la aplicación sobre la nomofobia, se aplica el cuestionario NMP-Q en dos momentos:
    \begin{itemize}
        \item \textit{Fase inicial (pre-test):} antes del uso de la aplicación.
        \item \textit{Fase final (post-test):} después de un periodo de uso definido.
    \end{itemize}
    Los resultados obtenidos permiten analizar los cambios en los niveles de nomofobia y en la adopción de hábitos de autorregulación, mediante la comparación de las mediciones iniciales y finales, con el fin de determinar el efecto de la aplicación móvil en los usuarios. \\

    \item \textbf{Documentación y sistematización} \\
    Se realiza la documentación detallada de todas las fases del proyecto, incluyendo las decisiones de diseño, los ajustes realizados y el análisis de los resultados obtenidos a partir del cuestionario NMP-Q. Esta documentación sirve como base para la redacción del proyecto de grado. \\

    \item \textbf{Sustentación} \\
    Finalmente, se presenta el proyecto de grado mediante la sustentación de los resultados obtenidos, las conclusiones alcanzadas y las recomendaciones para futuras mejoras o investigaciones relacionadas.
\end{enumerate}
