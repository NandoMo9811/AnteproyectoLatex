\begin{center}
\bfseries UNIVERSIDAD DEL CAUCA \\ 
\vspace{0.3cm}
 FACULTAD DE INGNERÍA ELECTRÓNICA Y TELECOMUNICACIONES \\
 \vspace{0.3cm} 
 ACTA DE ACUERDO SOBRE LA PROPIEDAD INTECLECTUAL DEL TRABAJO DE GRADO
\end{center}

En atención al acuerdo del Honorable Consejo Superior de la Universidad del Cauca, número 008 del 23 de febrero de 1999, donde se estipula todo lo concerniente a la producción intelectual en la institución, los abajo firmantes, reunidos el día del mes de  Enero del año 2026 en el salón del Consejo de Facultad, acordamos las siguientes condiciones para el desarrollo y posible usufructo del siguiente trabajo:

\textbf{Material del Acuerdo:} Trabajo de grado para optar al título de Ingeniero en Electrónica y Telecomunicaciones.

\textbf{Título del Proyecto:} Diseño e implementación de una aplicación móvil con retroalimentación háptica y sonora para la reducción de la nomofobia en jóvenes adultos.

\textbf{Objetivo del Proyecto:} Diseñar e implementar una aplicación móvil con retroalimentación háptica y sonora orientada a la reducción de los niveles de nomofobia en jóvenes adultos, promoviendo un uso más consciente y equilibrado del smartphone.

\textbf{Duración del Proyecto:} 9 meses.

Los participantes del proyecto, los señores estudiantes de pregrado MARIA JOSÉ CABRERA PANTOJA y FERNANDO MOLINA PLAZA, identificados con las cédulas de ciudadanía número 10.007.522.207 de Popayán y 1.152.714.291 de Medellín respectivamente, a quienes en adelante se les llamará “ESTUDIANTES”, el ingeniero CÉSAR ALBERTO COLLAZOS ORDÓÑEZ en calidad de Director del trabajo de grado, identificado con la cédula de ciudadanía número 76.309.486, y el ingeniero ŁUKASZ TOMCZYK en calidad de Codirector del trabajo de grado, identificado con el pasaporte EN3829009 y lugar Polonia, a quienes en adelante se les llamará “DOCENTES”, y la Universidad del Cauca, representada por el Decano de la FIET, manifiestan que:

\begin{enumerate}
    \item La idea original del proyecto es de Maria José Cabrera Pantoja y Fernando Molina Plaza, quien la propuso y presentó al Departamento de Sistemas, que la aceptó como tema para el proyecto de grado en referencia. \\
    \item La idea mencionada fue acogida por los estudiantes como proyecto para obtener el grado de Ingeniero en Electrónica y Telecomunicaciones, quienes la desarrollarán bajo la dirección del docente. \\
    \item Los derechos intelectuales y morales corresponden al docente y a los estudiantes. \\
    \item Los derechos patrimoniales corresponden al docente, a los estudiantes y a la Universidad del Cauca por partes iguales y continuarán vigentes, aún después de la desvinculación de alguna de las partes de la Universidad. \\
    \item Los participantes se comprometen a cumplir con todas las condiciones de tiempo, recursos, infraestructura, dirección y asesoría establecidas en el anteproyecto, a estudiar, analizar, documentar y hacer acta de cambios aprobados por el Consejo de Facultad durante el desarrollo del proyecto, los cuales entran a formar parte de las condiciones generales. \\
    \item Los estudiantes se comprometen a restituir en efectivo y de manera inmediata a la Universidad los aportes recibidos y los pagos hechos por la Institución a terceros por servicios o equipos, si el Comité de Investigaciones declara suspendido el proyecto por incumplimiento del cronograma o de las demás obligaciones contraídas por los estudiantes; y en cualquier caso de suspensión, la obligación de devolver en el estado en que les fueron proporcionados y de manera inmediata los equipos de laboratorio, de cómputo y demás bienes suministrados por la Universidad para la realización del proyecto. \\
    \item El docente y los estudiantes se comprometen a dar crédito a la Universidad y hacer mención del Fondo de Fomento de Investigación en los informes de avance y de resultados, y en registro de estos, cuando ha habido financiación de la Universidad o del Fondo. \\
    \item Cuando por razones de incumplimiento, legalmente comprobadas, de las condiciones de desarrollo planteadas en el anteproyecto y sus modificaciones, alguno de los participantes deba ser excluido del proyecto, los derechos aquí establecidos concluyen para él. Además, se tendrán en cuenta los principios establecidos en el reglamento estudiantil vigente de la Universidad del Cauca en lo concerniente a la cancelación y la pérdida del derecho a continuar estudios. \\
    \item El documento del anteproyecto y las actas de modificaciones, si las hubiere, forman parte integral de la presente acta. \\
    \item Los aspectos no contemplados en la presente acta serán definidos en los términos del acuerdo 008 del 23 de febrero de 1999 expedido por el Consejo Superior de la Universidad del Cauca, del cual los participantes aseguran tener pleno conocimiento. \\
\end{enumerate}
    
\vspace{1.5cm}

\begin{center}
    \begin{tabular}{c c}
        \rule{6.5cm}{0.4pt} & \rule{6.5cm}{0.4pt} \\
        César Alberto Collazos Ordóñez & Łukasz Tomczyk \\
        \textbf{Director} & \textbf{Codirector} \\
    \end{tabular}
\end{center}

\vspace{1cm}

\begin{center}
    \begin{tabular}{c c}
        \rule{6.5cm}{0.4pt} & \rule{6.5cm}{0.4pt} \\
        Maria José Cabrera Pantoja & Fernando Molina Plaza \\
        \textbf{Estudiante} & \textbf{Estudiante} \\
    \end{tabular}
\end{center}

\vspace{1cm}

\begin{center}
    \rule{6.5cm}{0.4pt} \\
    Alejandro Toledo Tovar \\
    \textbf{Decano FIET}
\end{center}



