\subsection{Objetivo General}

Desarrollar una aplicación móvil con retroalimentación háptica y sonora orientada a la reducción de los niveles de nomofobia en jóvenes adultos, promoviendo un uso más consciente y equilibrado del smartphone.

\subsection{Objetivos Especificos}

\begin{itemize}
    \item Caracterizar los patrones de uso del smartphone en jóvenes adultos entre 18 y 26 años, evaluando sus niveles de nomofobia mediante el cuestionario validado NMP-Q, con el fin de identificar hábitos de uso y factores de riesgo asociados al uso problemático del dispositivo.
    \item Desarrollar una aplicación móvil que integre el monitoreo del uso del smartphone y mecanismos de retroalimentación háptica y sonora, orientados a la autorregulación y a la reducción de conductas de uso problemático del dispositivo.
    \item Evaluar la efectividad de la aplicación móvil mediante pruebas con usuarios, comparando los niveles de nomofobia antes y después de la intervención a través del cuestionario NMP-Q.
\end{itemize}