Se presenta el presupuesto del proyecto distribuido por categorías y fuentes de financiación. La Tabla \ref{tab:presupuesto} fue elaborada siguiendo los lineamientos establecidos por la Facultad de Ingeniería Electrónica y Telecomunicaciones de la Universidad del Cauca para la formulación de presupuestos en anteproyectos \cite{UnicaucaGuiaAnteproyecto}, considerando un valor de \$22.358 COP por punto para el año 2025. El presupuesto corresponde a una duración total del proyecto de nueve (9) meses, equivalentes a treinta y seis (36) semanas de trabajo.

Para la estimación de los costos se tuvieron en cuenta los siguientes criterios:

\begin{itemize}
    \item \textbf{Recursos Humanos:} \\
    La dirección del proyecto estará a cargo de un director y un codirector, a quienes se les asigna una dedicación de 2 horas semanales, con un valor de 2.5 puntos por hora, durante un periodo de 36 semanas, conforme a la normativa institucional.

    La ejecución del proyecto estará a cargo de dos estudiantes, a quienes se les asigna una dedicación de 30 horas semanales por estudiante, con un valor de 1.5 puntos por hora, durante 36 semanas. Esta dedicación corresponde al aporte académico requerido para el desarrollo del trabajo de grado. \\

    \item \textbf{Recursos Técnicos:} \\
    Se considera el uso de equipos de cómputo personales, cuyo costo se estima como el 33\% del valor comercial actual, así como el uso de dispositivos móviles inteligentes con sistema operativo Android, necesarios para el desarrollo, implementación y validación de la aplicación móvil propuesta. \\

    \item \textbf{Recursos Software:} \\
    Se contempla el uso de servicios de bases de datos en la nube, específicamente Google Cloud Firestore, considerando un 20\% del costo estimado de adquisición, correspondiente a un escenario de uso parcial durante las fases de desarrollo y prueba del sistema. Asimismo, se incluyen herramientas de software para el desarrollo y diseño de la aplicación. \\

    \item \textbf{Recursos Bibliográficos:} \\
    Se estima un costo asociado al acceso a documentación científica especializada, correspondiente a la consulta de artículos y publicaciones académicas no cubiertas por las licencias institucionales. \\

    \item \textbf{A.U.I (Administración, Utilidades e Imprevistos):} \\
    El rubro de Administración, Utilidad e Imprevistos (A.U.I.) corresponde al 20\% del costo total del proyecto, e incluye los gastos relacionados con infraestructura física, gestión administrativa y una reserva para posibles imprevistos durante el desarrollo del proyecto.
\end{itemize}

\begin{table}[ht]
\centering
\caption{Presupuesto del proyecto}
\label{tab:presupuesto}
\renewcommand{\arraystretch}{1.25}
\begin{tabular}{p{4.2cm}ccc}
\toprule
\multirow{2}{*}{\textbf{Rubros}} & \multicolumn{2}{c}{\textbf{Fuentes}} & \multirow{2}{*}{\textbf{Total}} \\
\cmidrule(lr){2-3}
 & \textbf{Estudiantes} & \textbf{Departamento} &  \\
\midrule

\multicolumn{4}{l}{\textbf{Recursos Humanos}} \\
Director & -- & \$4'024.440 & \$4'024.440 \\
Codirector & -- & \$4'024.440 & \$4'024.440 \\
Estudiantes (2) & \$72'439.920 & -- & \$72'439.920 \\
\addlinespace

\multicolumn{4}{l}{\textbf{Recursos Técnicos}} \\
\textit{Recursos Hardware} \\
Utilización de PC & \$1.650.000 & -- & \$1.650.000 \\
Utilización de Smartphone & \$660.000 & -- & \$660.000 \\
Otros & \$300.000 & -- & \$300.000 \\
\addlinespace

\textit{Recursos Software} \\
Servicios de Google Cloud Firestore & \$360.000 & -- & \$360.000 \\
Herramientas Software & \$150.000 & -- & \$150.000 \\
\addlinespace

\multicolumn{4}{l}{\textbf{Recursos Bibliográficos}} \\
Documentación & \$250.000 & -- & \$250.000 \\
\midrule

Sub Total & \$75'809.920 & \$8'048.880 & \$83'858.800 \\
A.U.I (20\%) & \$15'161.984 & \$1'609.776 & \$16'771.760 \\
\midrule
\textbf{Total} & \textbf{\$90'971.904} & \textbf{\$9'658.656} & \textbf{\$100'630.560} \\
\bottomrule
\end{tabular}
\end{table}
