En la actualidad, el uso de los teléfonos inteligentes se ha consolidado como un elemento 
esencial en la vida cotidiana, al facilitar la comunicación, el acceso a la información, el entretenimiento y el desarrollo de actividades tanto acádemicas como laboral. No obstante, diversos estudios han evidenciado que el incremento sostenido en el tiempo de uso de smartphones ha dado lugar a hábitos tecnológicos poco saludables, especialmente en jóvenes adultos y estudiantes universitarios  \cite{ELHAI2017251}, \cite{MontagWalla2016}. Esta población presenta una alta exposición diaria a los dispositivos electrónicos, lo cual plantea desafíos significativos en términos de bienestar y autorregulación del uso de la tecnología.

La elevada disponibilidad y portabilidad del smartphone, junto con la necesidad constante de interacción digital, han favorecido el incremento progresivo del tiempo de uso del dispositivo entre los jóvenes adultos. Investigaciones recientes evidencian que esta población dedica, en promedio, entre 4 y 6 horas diarias al teléfono móvil, consolidando patrones de exposición prolongada a lo largo del día \cite{Andrews2015}. Este uso intensivo no solo incrementa la dependencia hacia el dispositivo, sino que también interfiere en la organización de las actividades cotidianas y en la capacidad de autorregulación del tiempo de uso, configurando un escenario propicio para la aparición de conductas problemáticas asociadas al teléfono inteligente. 

Adicionalmente, se ha identificado que el uso del smartphone durante horas previas al descanso es una práctica ampliamente extendida entre los jóvenes adultos. Estudios recientes reportan que una proporción significativa de estudiantes universitarios utiliza el teléfono móvil antes de dormir, lo cual se asocia con una disminución en la calidad y duración del sueño \cite{Joshi2021}. Investigaciones basadas en mediciones objetivas del uso del teléfono móvil han demostrado que los individuos con mayores niveles de uso presentan una reducción significativa en el tiempo total del sueño y una mayor probabilidad de experimentar problemas relacionados con el descanso nocturno \cite{Lemola2015}.

En este contexto emerge la nomofobia, definida como el miedo o ansiedad experimentados ante la imposibilidad de utilizar el teléfono móvil o de mantenerse conectado \cite{YildirimCorreia2015}. Estudios realizados mediante instrumentos válidos como la Nomophobia Questionnaire (NMP-Q) indican que entre el 30\% y el 50\% de los jóvenes adultos presentan niveles moderados o severos de nomofobia, lo que evidencia una alta prevalencia de este fenómeno en entornos universitarios \cite{YildirimCorreia2015},\cite{LeonMejia2021}. Estos resultados confirman que la nomofobia constituye una problemática relevante asociada a patrones de dependencia tecnológica. 

Diversas investigaciones han evidenciado que el uso excesivo del smartphone y la dependencia hacia este dispositivo se asocian con múltiples consecuencias negativas en la salud y el bienestar en los jóvenes adultos. En particular, se ha identificado una relación significativa entre el uso prolongado del teléfono móvil y el incremento de síntomas de ansiedad, estrés y dificultades en la regulación emocional \cite{ELHAI2020105962}, \cite{healthcare11142066}. Así mismo, investigaciones previas han demostrado que la utilización frecuente del smartphone, especialmente durante horas nocturnas, se asocia con alteraciones en la calidad y duración del sueño, lo que repercute negativamente en el descanso y en el funcionamiento diario de los estudiantes universitarios \cite{EXELMANS201693}. Estas afectaciones pueden derivar, a su vez, en una disminución del rendimiento académico, problemas de concentración y deterioro de las relaciones sociales, en la calidad de los jóvenes adultos \cite{SAMAHA2016321}, \cite{LEPP2014343}.

A pesar del creciente interés académico en el estudio de la nomofobia y del uso problemático del smartphone, la mayoría de las estrategias reportadas se centran en enfoques educativos o intervenciones psicológicas tradicionales. Si bien existen aplicaciones móviles orientadas a limitar el tiempo de uso del dispositivo, estas suelen basarse en mecanismos de control externo que pueden afectar la aceptación y adherencia por parte de los usuarios \cite{vanVelthoven2018}. En particular, se identifica una limitada disponibilidad de soluciones tecnológicas que integren estrategias de autorregulación apoyadas en retroalimentación háptica y sonora como mecanismo orientado a la reducción frente al uso problemático del smartphone \cite{Hampton2025}.

Dadas las circunstancias descritas, es posible observar que la nomofobia actúa como una telaraña invisible que atrapa progresivamente a los jóvenes adultos, condicionando sus hábitos cotidianos y su relación con el smartphone. En este escenario, se evidencia la necesidad de diseñar e implementar una aplicación móvil orientada a la reducción de la nomofobia, que aproveche las capacidades sensoriales del propio dispositivo para generar alertas no intrusivas y promover hábitos de uso más conscientes y equilibrados. La incorporación de retroalimentación háptica y sonora permite proponer una solución tecnológica innovadora, alineada con los principios del bienestar digital y con las tendencias actuales en el desarrollo de aplicaciones centradas en el usuario, contribuyendo a la mitigación de los efectos asociados al uso excesivo del teléfono inteligente.

En este contexto, surge la siguiente pregunta de investigación: 

\textit{¿Cómo puede una aplicación móvil con retroalimentación háptica y sonora contribuir a la reducción de los niveles de nomofobia, medidos mediante el Nomophobia Questionnaire (NMP-Q), en jóvenes adultos?}

\textbf{Hipótesis} 

Se plantea que el uso de una aplicación móvil con retroalimentación háptica y sonora contribuye significativamente a la reducción de los niveles de nomofobia, medidos mediante el Nomophobia Questionnaire (NMP-Q), en jóvenes de 18 a 26 años de edad, favoreciendo procesos de autorregulación del uso del smartphone y la adopción de hábitos tecnológicos más saludables.

Como hipótesis nula, se considera que la utilización de dicha aplicación móvil no produce cambios significativos en los niveles de nomofobia, medidos mediante el NMP-Q, en jóvenes adultos.
