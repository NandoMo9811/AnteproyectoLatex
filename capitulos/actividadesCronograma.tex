\subsection{Actividades}
El proyecto se desarrollará durante un periodo de nueve meses, siguiendo la metodología de Diseño Centrado en el Usuario, estructurada en siete fases secuenciales e iterativas, cada una con un propósito específico.

\textbf{Actividad 1. Análisis del usuario y del contexto}

Se realiza una revisión bibliográfica y un diagnóstico inicial mediante la aplicación del cuestionario NMP-Q, con el fin de identificar los niveles de nomofobia y los patrones de uso del smartphone en los participantes. Esta actividad permite comprender las necesidades reales de los usuarios.

\textbf{Actividad 2. Definición de requisitos}

Con base en la información recolectada, se establecen los requisitos funcionales y no funcionales de la aplicación, definiendo las características técnicas y de experiencia de usuario que debe cumplir la solución propuesta.

\textbf{Actividad 3. Diseño de la solución}

Se diseñan las interfaces y se elaboran los prototipos de la aplicación móvil, priorizando la usabilidad, la claridad y la eficiencia. El diseño busca facilitar una interacción intuitiva que favorezca el uso consciente y autorregulado del smartphone.

\textbf{Actividad 4. Desarrollo de la aplicación:}

Se implementan las funcionalidades principales de la aplicación, incluyendo el cuestionario NMP-Q, el monitoreo del uso del smartphone y los mecanismos de retroalimentación háptica y sonora. Asimismo, se realizan pruebas funcionales para garantizar su correcto desempeño.

\textbf{Actividad 5. Evaluación con usuarios:}

Se lleva a cabo la evaluación del sistema mediante pruebas de usabilidad durante un periodo de uso controlado, analizando aspectos como la facilidad de uso, la comprensión de las funciones, la eficiencia en la realización de tareas y el nivel de satisfacción del usuario.

\textbf{Actividad 6. Ajustes y optimización:}

Con base en los resultados de la evaluación, se realizan ajustes en la interfaz y en las funcionalidades de la aplicación para mejorar su desempeño y experiencia de uso.

\textbf{Actividad 7. Documentación y cierre:}

Se documenta todo el proceso de desarrollo, se analizan los resultados finales y se elabora el informe del proyecto de grado para su respectiva sustentación.

\subsection{Cronograma de Actividades}

\begin{figure}[H]
    \centering
    \includegraphics[width=0.95\textwidth]{figuras/CronogramaActividades.png}
    \caption{Cronograma de Actividades.}
    \label{fig::CronogramaActividades}
\end{figure}
