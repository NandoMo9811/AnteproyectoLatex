
{
    \footnotesize
    \renewcommand{\arraystretch}{1.1}

    \begin{longtable}{c p{3cm} p{3cm} p{3cm} p{3cm}}
        
        \caption{\centering{Resumen de Trabajos Relacionados.}}
        \label{tab::estadoArteNomofobia} \\

        \hline
        
        \textbf{Ref} & \textbf{Artículo} & \textbf{Tipo de Intervención y Enfoque} & \textbf{Principales Aportes} & \textbf{Limitaciones Identificadas} \\ \hline
        \endfirsthead

        \cite{CelikKucukKulaberoglu2024} & Family Supported Nomophobia Reduction Intervention for Adolescents & Intervención psicoeducativa presencial con enfoque en educación digital, autorregulación y acompañamiento familiar & Reducción significativa y sostenida de la nomofobia; evidencia el impacto positivo del entorno familiar & Limitada al contexto presencial; baja escalabilidad y ausencia de soporte tecnológico continuo \\ 

        \hline

        \cite{AiniBukhoriBakar2021} & The Role of Mindfulness and Digital Detox to Adolescent Nomophobia & Intervención conductual presencial basada en mindfulness y desconexión digital voluntaria & Disminución de nomofobia y ansiedad; refuerza la conciencia sobre el uso del smartphone & No integra herramientas tecnológicas ni seguimiento fuera del entorno terapéutico \\

        \hline

        \cite{Keller2021} & Not Less But Better & Aplicación móvil orientada a autorregulación, planificación del uso y fortalecimiento de la autoeficacia &Reducción del uso problemático del smartphone mediante metas y reflexión & No está dirigida específicamente a la nomofobia ni incorpora retroalimentación sensorial \\

        \hline

        \cite{ALOS2024102900} & Mobile health in primary care & Soluciones de mHealth orientadas a promoción del bienestar y personalización de intervenciones & Resalta el potencial del mHealth para empoderar usuarios y promover hábitos saludables & Señala desafíos de validación clínica, usabilidad y sostenibilidad \\

        \hline

        \cite{CunhaSouzaSantiago2020} & Girassol & Aplicación móvil para medición y detección temprana de nomofobia mediante cuestionarios validados & Alta aceptación y utilidad para identificar patrones de riesgo & Enfoque limitado al diagnóstico; no incluye intervención activa \\

        \hline

        \cite{KoYangLee2015} & NUGU & Aplicación móvil grupal basada en apoyo social y autorregulación colectiva &Evidencia que la dinámica social favorece la reducción del uso excesivo del smartphone & No aborda explícitamente la nomofobia ni el componente emocional \\

        \hline

        \cite{Hiniker2016} & MyTime & Aplicación de autorregulación basada en metas personales y control voluntario del uso & Reducción del uso improductivo sin afectar actividades valoradas & No considera ansiedad por desconexión ni retroalimentación sensorial \\

        \hline

        \cite{FoulonneauCalvaryVillain2016} & TILT & Aplicación persuasiva con auto-monitoreo y mensajes adaptativos según contexto & Reducción sostenida del tiempo de uso mediante persuasión contextual & Enfoque centrado en control conductual, no en nomofobia \\

        \hline

        \cite{LoechtefeldBoehmerGanev2013} & AppDetox & Aplicación de autorregulación basada en reglas autoimpuestas y restricción de apps & Evidencia efectividad del autocontrol, especialmente sobre redes sociales & Basada en bloqueo; no aborda ansiedad ni procesos emocionales \\

        \hline

        \cite{Neuwirth2020} & Flipd & Aplicación educativa para limitar el uso del smartphone en el aula mediante bloqueo temporal & Mejora asistencia y participación académica & Aplicación restringida al contexto educativo; ignora la dimensión emocional \\

        \hline
    \end{longtable}
}